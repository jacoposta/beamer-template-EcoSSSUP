\documentclass[english]{beamer}
\usepackage{ae}

%%%%%%%%%%%%%%%%%%%%%%%%%%%%%%%%%%%%%%%%%%%%%%%%%%%%%%%%%
% Begin of Institute of Economics specific style macros %
%%%%%%%%%%%%%%%%%%%%%%%%%%%%%%%%%%%%%%%%%%%%%%%%%%%%%%%%%

\usepackage{tikz}

\setbeamertemplate{title page}[default][left]

\setbeamertemplate{background}{
	\begin{tikzpicture}
		\useasboundingbox (0,0) rectangle(\the\paperwidth,\the\paperheight);
		\ifnum\value{page}=1\relax
			\node[anchor=south west,inner sep=0] at (0.7,7) {\includegraphics[width=0.35\paperwidth]{ecosssup_logo}};
			\node[anchor=center,inner sep=0] at (13,4) {\includegraphics[width=0.5\paperwidth]{school_watermark}};
		\fi
	\end{tikzpicture}
}

\setbeamercolor{footlinetext}{bg=, fg=white}

\defbeamertemplate*{footline}{}{
	\leavevmode%
	\begin{tikzpicture}
		\useasboundingbox (0,0) rectangle(\the\paperwidth,1);
		\ifnum\value{page}>1\relax%	
			\fill[color=gray!80] (0,0) rectangle (\the\paperwidth,1);
			\node[anchor=north west,inner sep=0] at (11,1.5) {\includegraphics[width=0.26\paperwidth]{school_white}};
			\node[right] at (0.25,0.7) {\scriptsize
			\begin{beamercolorbox}[wd=.8\paperwidth,ht=2.25ex,dp=1ex,left]{footlinetext}
			\usebeamerfont{footlinetext}\insertshorttitle
			\end{beamercolorbox}
			};  
			\node[right] at (0.25,0.25) {\scriptsize
			\begin{beamercolorbox}[wd=.8\paperwidth,ht=2.25ex,dp=1ex,left]{footlinetext}
			\usebeamerfont{footlinetext}\insertshortauthor\hfill\insertshortdate{}\hfill\insertframenumber{}/\inserttotalframenumber
			\end{beamercolorbox}
			};
		\fi
	\end{tikzpicture}
	\vskip0pt%
}

\setbeamertemplate{navigation symbols}{}

%%%%%%%%%%%%%%%%%%%%%%%%%%%%%%%%%%%%%%%%%%%%%%%%%%%%%%%
% End of Institute of Economics specific style macros %
%%%%%%%%%%%%%%%%%%%%%%%%%%%%%%%%%%%%%%%%%%%%%%%%%%%%%%%

\title[The EcoSSSUP \texttt{beamer} template]{The EcoSSSUP \texttt{beamer} template}
\author[Jacopo Staccioli]{Jacopo Staccioli}
\institute{\href{mailto:j.staccioli@santannapisa.it}{\texttt{j.staccioli<at>santannapisa.it}}}
\date[\today]{Pisa, \today}

\begin{document}

\begin{frame}
\vspace{15ex} % adjust the vertical offset if title covers the Institute logo
\titlepage
\end{frame}

\begin{frame}
\frametitle{Premise}
This document, together with the associated package, aims at providing a \LaTeX\ implementation of the \texttt{beamer} class fully consistent with the Institute of Economics official graphic design.\\
\bigskip
Customisation is purposefully introduced in terms of \texttt{beamer} style macros, rather than as a standalone \texttt{beamer} theme, to preserve the greatest deal of flexibility on the user side.
\end{frame}

\begin{frame}
\frametitle{Contents}
\tableofcontents
\end{frame}

\section{Structure and usage}

\begin{frame}
\frametitle{Structure and usage}
The package consists of five distinct files:
\begin{description}
\item[\texttt{ecosssup-template.pdf}]: \emph{this} very document
\item[\texttt{ecosssup-template.tex}]: the source code of \emph{this} document
\item[\texttt{ecosssup\_logo.pdf}]: the Institute logo in the \texttt{titlepage}
\item[\texttt{school\_watermark.pdf}]: the watermark in the \texttt{titlepage}
\item[\texttt{school\_white.pdf}]: the (partial) logo in the \texttt{footer}
\end{description}
\bigskip
In order to adopt the proposed macros it is sufficient to:
\begin{enumerate}
\item modify the \texttt{.tex} file as needed\footnote{otherwise, copy the specific file macros (highlighted) in your own \emph{preamble}}
\item enclose the last three \texttt{.pdf} files within the same directory
\item compile with your preferred \LaTeX\ engine
\end{enumerate}
\end{frame}

\section{Compatibility}

\begin{frame}
\frametitle{Compatibility}
The provided macros do \emph{not} constitute a theme and are neither theme-specific, nor style-specific, beyond what is hereby defined.\\
\bigskip 
In principle, they should be compatible with all the built-in \texttt{beamer} themes, as long as the theme is called \underline{\emph{before}} the macros are defined in the underlying \texttt{.tex} file preamble.\\
\bigskip
Feel free to choose your favourite \texttt{beamer} theme\footnote{The \texttt{default} theme is used in \emph{this} very presentation} but make sure the code complies with the aforementioned ordering.
\end{frame}

\section{Title page}

\begin{frame}[fragile]
\frametitle{Title page}
The \texttt{titlepage} is defined as a \texttt{plain}-style frame with the Institute logo in the north-west corner and a watermark half trespassing the eastern edge.\\
\bigskip
If necessary, adjust the \verb|\vspace{}| offset before calling \verb|\maketitle| or \verb|\titlepage| in order to prevent the writings from covering the Institute logo and/or the watermark.\\
\bigskip
CAUTION! Since the original setup is in 4:3 aspect ratio, any layout different from 4:3 is incompatible with the proposed macros.
\end{frame}

\section{Frame footer}

\begin{frame}
\frametitle{Frame footer}
Every frame other than the \texttt{titlepage} contains a bottom footer.\\
\bigskip
The footer consists of a gray rectangle bar and two overlapping containers:
\begin{itemize}
\item the first only shows the \texttt{shorttitle} variable\
\item the second encloses the \texttt{shortauthor} and the \texttt{shortdate} variables, and a \texttt{n/N} frame counter
\end{itemize}
\bigskip
CAUTION! Had the frame body been longer than available space, the exceeding part would just disappear behind the footer.
\end{frame}

\end{document}